\documentclass[xcolor=dvipsnames]{beamer}

\title{Concurrency in Clojure and Go}
\author{Stephen Adams}
\date{May 3, 2012}

% Setup appearance:
\usepackage{beamerthemesplit}
\usecolortheme[named=Brown]{structure} 
%\usetheme{Darmstadt}
%\usefonttheme[onlylarge]{structurebold}
%\setbeamerfont*{frametitle}{size=\normalsize,series=\bfseries}
%\setbeamertemplate{navigation symbols}{}


% Standard packages

\usepackage[english]{babel}
\usepackage[latin1]{inputenc}
\usepackage{times}
\usepackage[T1]{fontenc}
\usepackage[round]{natbib}


\begin{document}

\begin{frame}
	\titlepage
\end{frame}
\section*{outline}
\section{Introduction}

		\begin{frame}{Clojure}
                The Clojure programming language was released in 2007 by the software developer Rich Hickey. Clojure was designed with four features in mind:
                \pause
                \begin{block}{The Four Features of Clojure}
                \begin{itemize}
                        \item A Lisp
                        \item Functional programming
                        \item Symbiosis with an established platform (Java)
                        \item Designed for concurrency
                \end{itemize}
                \end{block}
                
         \begin{frame}{Go}
         	Go was first made available in 2009 by Google Inc. after two years of in house development. The primary goal of Go is to
         	
         	``provide the efficiency of a statically-typed compiled language with the ease of programming of a dynamic language" additionally goals of Go include:
         	\pause
         	\begin{block}{Goals for Go}
         		\begin{itemize}
         			\item Type and memory safety
         			\item Good concurrency support
         			\item Efficient and latency-free garbage collection
         			\item High-speed compilation
         		\end{itemize}
         	\end{block}
         \end{frame}	
\end{document}