\documentclass{beamer}

\title{Brain Computer Interfaces}
\author{Stephen Adams}
\date{December 4th, 2010}

% Setup appearance:

\usetheme{Darmstadt}
\usefonttheme[onlylarge]{structurebold}
\setbeamerfont*{frametitle}{size=\normalsize,series=\bfseries}
\setbeamertemplate{navigation symbols}{}


% Standard packages

\usepackage[english]{babel}
\usepackage[latin1]{inputenc}
\usepackage{times}
\usepackage[T1]{fontenc}


% Setup TikZ

\usepackage{tikz}
\usetikzlibrary{arrows}
\tikzstyle{block}=[draw opacity=0.7,line width=1.4cm]

\begin{document}

\begin{frame}
	\titlepage
\end{frame}

\section*{Outline}
	\begin{frame}
		\tableofcontents
	\end{frame}
	
	\section{Introduction}
	
	\section{EEG}
		\begin{frame}
			\frametitle{Event Related Potentials}
			 \begin{block}{Definition}
				 An event related potential (ERP) is a voltage reading over time from a certain are of the brain. These voltages are fired by the brain due to a certain thought or perception.
			 \end{block}			 
			 \begin{itemize}
			 \item The brain is constantly firing ERPs
			 \item Only some of these ERPs have had their sources and meanings discovered.
			 \item ERPs are usually named with a P or a V followed by a number. The letter is for positive or negative voltage and the number is the latency.
			 \end{itemize}
		\end{frame}
	\section{BCI Calibration and Classification}
	
	\section{Uses of BCI systems}
	
	\section{Conclusion}	
	
\end{document}