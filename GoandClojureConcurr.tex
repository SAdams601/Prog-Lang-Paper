\documentclass{acm_proc_article-sp}
\usepackage{url}
%\usepackage{hyperref}
\begin{document}

\title{A comparison of the concurrency features available in the Go and Clojure programming languages}

\numberofauthors{1} 
\author{
\alignauthor
Stephen Adams\\*
       \affaddr{University of Minnesota Morris}\\
       \affaddr{600 E 4th St.}\\
       \affaddr{Morris, Minnesota USA}\\
       \email{adams601@morris.umn.edu}
}

\maketitle
\begin{abstract}
	In a 2005 interview Gordon Moore predicted that his namesake law (Moore's law states that the ``number of transistors that can be placed inexpensively on an integrated circuit doubles approximately every two years") will reach a fundamental limit in ten to twenty years \cite{wiki:moore}. Our transistors are starting to approach the size of individual atoms which is a boundary that will not be broken. With this in mind, the challenge to continue increase the speed of computing will soon fall to computer scientists as our abilities to create programs that utilize multiple processing cores will be where the futures speed gains are created.
	
	In the past five years two languages were created (Go and Clojure)\cite{wiki:timeline} with support for easy concurrency being a primary tenet of these languages. Both of these languages have received significant levels of interest (both languages are in the top 100 programming languages in the Tiobe index \cite{tiobe}) to think that they may either see large scale adoption for their concurrency features or influence the next generation of concurrency control methods.
In this paper I will briefly cover the basics of both Go and Clojure syntax and then I will cover these two languages' concurrency features. Finally there will be a discussion about the merits of both of these concurrency systems.
\end{abstract}

\keywords{Concurrency, Programming Languages, Software Transactional Memory}

\section{Introduction}
	"A big challenge for concurrent programs is managing mutable state. If mutable state can be accessed concurrently, then as a programmer you must be careful to protect that access."\cite{halloway} This quote by Stuart Halloway succinctly states why the traditional model of concurrency control is so difficult to manage properly, and consequently why it's seldom implemented. Clojure and Go both provide alternative, and different, concurrency control. Clojure through its software transactional memory\cite{wiki:clojure} and Go provides goroutines which are a lightweight thread\cite{wiki:go}. 
	
	This paper will first provide a brief overview of Clojure then the Go programming languages. Then specifically the concurrency features of these languages will be covered more in depth. Finally a comparison of the two concurrency methods and a brief conclusion.
	
\section{Language Basics}
	In this section the very basics of Clojure and Go will be explained. The idea here is to provide the barest minimum of syntax knowledge so that code examples can be understood. If this section is not adequate or the reader is further interested in one of these languages \url{http://clojure.org} or \url{http://golang.org/} should provide plenty of information concerning their respective languages.
	
\subsection{An Introduction to Clojure}
	The first thing that becomes evident when looking at Clojure code is that it is a Lisps and therefore uses prefix notation like so:
	\begin{verbatim}
	(+ 2 3)
	\end{verbatim}
	Which returns the value 5. Clojure also supports functional language features such as anonymous functions and first class functions. Anonymous functions are created with the "fn" macro with is a little different from the more typical "lambda" function. When defining standard Clojure functions the "defn" macro is used. A macro is a textual transformation. Essentially a macro expands from one piece of code to another before any evaluation happens. Support for macros is an important feature in Lisp and it's dialects.

	Clojure also has the important concept of Clojure code is also Clojure data. Clojure has several data structures, the three most common of these structures are illustrated in table~\ref{coll:table}. As you can see in table~\ref{coll:table} each collection is denoted in Clojure by a different symbol literal. Clojure code itself is contained in these data structures. The code below shows a square function being defined, the parenthesis that everything is wrapped in is actually a list, and the third element in this list is a vector that contains the names for all of the parameters that the new square function takes in.
	
	\begin{verbatim}
	(defn square [x]
	   (* x x))
	\end{verbatim}
	
	\begin{table}[t]
	\caption{Clojure collection literals\label{coll:table}}	
	\begin{tabular}{ | l | c | }
	\hline
	Collection Type & Literal \\ \hline
	List & (1 2 3 4) \\ \hline
	Vector & ["apple" "banana" "orange"] \\ \hline
	Hashmap & \{ :fName "Stephen" :lName "Adams" \}\\
	\hline
	\end{tabular}
	\end{table}
	
	This has been a very brief introduction to Clojure syntax. As I mentioned before if you require more information \url{http://clojure.org} will provide a good starting point for further information. Now I will provide a brief introduction to the Go language.
	\subsection{Enough about Go to be dangerous}
	Work on Go began at Google Inc in 2007 \cite{wiki:go}. Go is a compiled for efficiency, but also provides more the ease of programming that a dynamically typed language provides \cite{wiki:go}. At first glance Go syntax resembles C code but types need only be declared when you do not initialize a variable when you declare it, for instance this will throw an error:
	
	\begin{verbatim}
	var t
	\end{verbatim}
	but both of these examples will compile and run:
	\begin{verbatim}
	var t int
	var x = 12
	\end{verbatim}
	Also unlike C, Go does not require users to write in semicolons instead the semicolons are automatically inserted~\cite{go:effective}.
	
	Overall go is a fully featured systems language that also provides many new features that systems languages haven't had before. The addition of garbage collection and the concurrency features covered in the next section are particularly unusual for a ``low" level language.
	
	The last two parts have been very quick and I would suggest you look into further documentation, both Clojure and Go have free online resources that you should consult for the following parts of the paper if you are unclear what some of the examples.
	
	\section{Concurrency features in Clojure and go}
	This section will cover the concurrency features of these two languages. Starting with Clojure's software transactional memory.
	
	\subsection{Clojure and Software Transactional Memory}
	One of the main ideas behind Lisps (and therefore Clojure) is that data is immutable. Data being immutable allows for Clojure to operate concurrently with a backdrop of safety so to speak. Without having to worry about data changing at all functions don't have to worry about other functions that are running in parallel changing data that you are dependent on. However, as handy as immutability is parallel functions still need to share data at some point and for that Clojure provides an implementation of a software transactional memory system (STM).
	
	The basic idea behind an STM is that a thread can ignore what every other is doing and just modify pieces of shared memory without problems. This is an optimistic view of concurrency that STMs take~\cite{wiki:stm}. Using the STM is based around transactions centered around shared pieces of memory. These pieces of memory are the only mutable data types in Clojure. Clojure supports a total of four different types of mutable data refs, agents, atoms, and vars~\cite{joy}.
	
	\begin{table}[b]
	\caption{Clojure Concurrent Reference Types\label{types:table}}	
	\begin{tabular}{ | l | c | c | c | c | }
	\hline
	& Ref & Agent & Atom & Var\\
	Coordinated & \checkmark & & &\\
	Asynchronous & & \checkmark & &\\
	Retriable & \checkmark & & \checkmark &\\
	Thread-local & & & & \checkmark\\
	\hline
	\end{tabular}
	\end{table}
	
	Table~\ref{types:table} shows all four of these types and which features they have. Coordinated refers to the quality that reads and writes to multiple types can be made with a guarantee of no race conditions. The asynchronous quality means that a request to update the type is queued to happen in another thread later, while the requesting thread continues immediately. Retriable indicates that any work that updates a value is speculative and may have to be repeated. Finally thread-local means that changes are isolated to a single thread, which achieves thread safety.
	
	The overview of Clojure concurrency will begin with the backbone of the four mutable types the ref~\cite{halloway}. The ref is a mutable storage spot for an immutable object. A new reference is created with the ``ref" function, which takes in the object to be stored in the . The contents of a ref can be read with the ``deref" function, deref also has a shortcut the ``\@" symbol.
	
	\begin{verbatim}
	;;Create a new reference and store it in a variable
	(def ref-test (ref [1, 2, 3, 4, 5]))
	;;Get the vector back
	(deref ref-test)
	@ref-test
	;;Both of these return [1, 2, 3, 4, 5]
	\end{verbatim}
	
	Of course the point of refs is that they are a mutable type and can change. To store a new value in a ref use the function ``ref-set" but since refs are mutable you must protect the updates. In many languages this is where a lock would be involved but instead Clojure will use a transaction~\cite{halloway}. Create a new transaction with the ``dosync" function like so:
	
	\begin{verbatim}
	;;Following from the code above
	(dosync (ref-set ref-test [1, 2, 3, 4]))
	@ref-test
	;;This now will return [1, 2, 3, 4]
	\end{verbatim}
	
	Clojure transactions provide some important guarantees about their behavior; they are:
	\begin{itemize}
	\item atomic
	\item consistent
	\item isolated
	\end{itemize}
	
	Atomic transactions means that if more than one ref is changed in a single transaction the effects of these updates will appear as a single event to parts of the program outside of the transaction. Consistency means that if any functions in a transaction fail the entire transaction will fail. Essentially transactions are all or nothing, either everything will happen or nothing will happen. Finally isolation is that while a transaction is running it cannot see what other transactions are doing~\cite{halloway}.
	
	In general this is the basic picture of Clojure concurrency. Transactions can be dispatched with functions inside of one of the four mutable types and while the functions run program execution can continue. Each of the four other types has different functions for creating and setting their values. The particulars of these will not be enumerated only because the focus of the paper is not a comprehensive survey of Clojure's and Go's concurrency features but rather a comparison between the two. Essentially Clojure provides the STM as a way of having mutable data so that parts of a program can go off and change the values contained within separately, and these values are retrieved at a later time by the main body of the program. Clojure also provides the safety of transactions to protect the programmer from running into the same problems that locks and other ``traditional" concurrency control methods can create.
	
	\subsection{Goroutines}
		Go provides a light weight thread feature known as a goroutine. There are very few guarantees about how a goroutine will actually run. Starting a new goroutine is really simple all that you are required to do is just put ``go" before the function that you want to be run in a new goroutine, like in this example from~\cite{go:tour}:
		
		\begin{verbatim}
				
func say(s string) {
	for i := 0; i < 5; i++ {
		runtime.Gosched()
		fmt.Println(s)
	}
}

func main() {
	go say("world")
	say("hello")
}
		\end{verbatim}
		
		This code will print:
		
		\begin{verbatim}
		hello
		world
		hello
		world
		hello
		world
		hello
		world
		hello
		\end{verbatim}
		
		This example also illustrates the challenge that goroutines present. Hello was printed five times but world was only printed four times even though the say function should have looped five times. This happens due to when the call to Gosched is placed in the say function. Take for instance if you reverse the order of the print and the call to Gosched so that say now reads:
		
		\begin{verbatim}
func say(s string) {
	for i := 0; i < 5; i++ {
		fmt.Println(s)
		runtime.Gosched()
	}
}
		\end{verbatim}
		
		And now this will be printed: 
		
		\begin{verbatim}
		hello
		world
		hello	
		world
		hello
		world
		hello
		world
		hello
		world
		\end{verbatim}
		
		This is probably the behavior that a programmer would want to have happen but it is not entirely evident from the current Go resources. If we look at what Go's docs say about the behavior of Gosched all it says is: ``Gosched yields the processor, allowing other goroutines to run. It does not suspend the current goroutine, so execution resumes automatically." This description does not describe how or why the change in order matters. 
		
		Fortunately Go allows has a few other ways of dealing with goroutines. Tools like Gosched are much lower level than is necessary for most applications.
		
		\subsubsection{Channels}
		Rather than deal with the traditional concurrency controls that most languages (Go included) provide, Go also provides the idea of channels to make synchronization with different routines much simpler.
		
		Channels are just pathways to send and receive values between different threads. Unlike many other parts of Go channels much be given a declared type. Creating a new channel is quite simple you just call the build in make function.
		
		\begin{verbatim}
		testChannel := make(chan string)
		//testChannel is a channel that can send and receive strings
		\end{verbatim}
		
		Channels come with their very own operator the arrow or "<-". The arrow will either send data to a channel or retrieve data from it depending on which side of the channel variable it is place on.
		
		\begin{verbatim}
		testChannel := make(chan string)
		testString := "test"
		testChannel <- testString //Sending to a channel
		test := <- testChannel //Retrieving from a channel
		\end{verbatim}
		
		The basic idea of the operator is that the data flows in the direction of the arrow either into or out of the channel. Channels will handle most of the low level things for you like blocking and waiting for data to come down a channel. For example: 
		
		\begin{verbatim}
package main

import (
        "fmt"
        "time"
)

func passer(receiver chan string, send chan string) {
        tmp := <-receiver
        send <- tmp
}


func main() {
        send := make(chan string)
        receive := make(chan string)
        str := "Test string"
        go passer(send, receive)
        time.Sleep(500 * time.Millisecond)
        send <- str
        fmt.Println(<-receive)
}
		\end{verbatim}
		
		The above example has a function called passer which takes in two channels, reads a piece of data from one and then sends that piece of data to the other channel. In main the two channels that passer will take are created as well as the string that will eventually be passed to one of the channels. Then passer will be sent off with a call to the go command and the main will sleep for five seconds. While main sleeps passer will be waiting for data to come down the channel it knows as receiver. In most other languages trying to read from an empty data pipeline would throw an error but in Go this is just fine because the arrow operator handles all of the waiting for you. While main sleeps passer is just sitting at it's first line ready to receive data and send it down the other channel, and as soon as main wakes up everything can proceed.
		
\section{Discussion on the merits of each system}
	Now that both Clojure and Go have been introduced, as well as their methods of dealing with concurrency, this section will discuss the relative merits of the Clojure and Go ways of doing concurrency.
	
	Clojure's STM acts in very similar ways to the logic surrounding Go's channels, both of these systems provide automatic locking to shared resources and certain guarantees about when and how things run. Though Clojure made its job a little easier by making data immutable by default, without the worry that a different thread will change data that you are dependent on Clojure code can operate, for the most part, oblivious to any other code that is just happens to be running at the same time. Go on the other hand has to worry about all of the odd interactions that can happen when different threads want to change but, from the cursory exploration that went into writing this paper, Google seems to have covered all of its bases.
	
\section{Conclusion}
	ADD THINGS HERE

\bibliographystyle{abbrv}
\bibliography{concurrency}

\end{document}
